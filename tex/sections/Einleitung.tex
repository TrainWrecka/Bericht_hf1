\section{Einleitung}

Ein grosses Thema der Hochfrequenztechnik sind Antennen. Generell können Antennen als Leiter angesehen werden, welche identisch zur Leitungstheorie gelten. Jedoch gibt es viele verschiedene Ausführungen welche sich alle unterschiedlich verhalten. Während des Unterrichtes wurden keine Antennen angeschaut, weshalb sich mit diesem Bericht die Möglichkeit angeboten hat, diesen Teilbereich aufzuarbeiten. Hierfür wurden im Simulationsprogramm CST Messungen ausgewählter Antennen durchgeführt, welche sich vor allem auf die Direktivität der Antennen beziehen. Als Abschluss wurde noch eine Kleeblatt Antenne modelliert, mit welcher am Institut gearbeitet wird. Zu jeder Antenne wurde zuerst die Theorie behandelt und anschliessend die entsprechenden Simulationen durchgeführt. Die Theorie wurde nach dem Buch von K. Kark aufgearbeitet \cite{book}.