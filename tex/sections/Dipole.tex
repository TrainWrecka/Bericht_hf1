\section{Hertzscher Dipol}

Als erstes Thema wird der Hertzsche Dipol beschrieben, welcher das Modell der Antenne durch seine kurze Länge vereinfacht.

\subsection{Theorie}\label{sec:HerDip}

Der Hertzsche Dipol lässt sich als offener Schwingkreis interpretieren \cite{web2}. Seine Entstehung kann schematisch als Deformation eines geschlossenen Schwingkreises, wie in Abbildung \ref{fig:entstehung} zu sehen ist, beschrieben werden. Die Frequenz dieses Schwingkreises lautet:

\begin{equation}
f = \dfrac{1}{2 \pi \sqrt{LC}}.
\end{equation} 

Da der Hertzscher Dipol eine hochfrequente elektromagnetische Schwingung erzeugen soll, muss die Induktivität $ L $ und die Kapazität $ C $ klein sein. Sie lassen sich wie folgt berechnen:

\begin{align}
L &= \mu_0  \mu_r  n^{2} \dfrac{A}{l} \\
C &= \varepsilon_0  \varepsilon_r   \dfrac{A}{d}
\label{eq:LC}
\end{align}

Um die Induktivität der Spule zu minimieren, muss die Permeabilitätszahl, die Windungszahl $ n $ und die Spulenfläche $ A $ möglichst klein und die Spulenlänge $ l $ möglichst gross sein. Die Permeabilitätszahl kann verkleinert werden, indem eine Luftspule verwendet wird.\\

Die Kapazität des Kondensators wird verkleinert mittels einer minimalen Dielektrizitätszahl und einer kleinen Plattenfläche $ A $ sowie durch einen grossen Plattenabstand $ d $. Um die Dielektrizitätszahl zu senken wird das Dielektrikum Luft verwendet.\\

In der Abbildung \ref{fig:entstehung} ist die Minimierung der Induktivität und der Kapazität symbolisch darge-stellt. So entsteht aus einem geschlossenen Schwingkreis ein offener Schwingkreis. Das Resultat ist ein langer, dünner Stab. Die Frequenz des Schwingkreises kann nun nicht mehr über die Induktivität und der Kapazität verändert werden. Der Hertzscher Dipol lässt sich nur noch über eine Längenänderung anpassen.

\begin{figure}[H]
	\centering
	\includegraphics[width=0.75\linewidth]{entstehung}
	\caption{Entstehung eines Hertzschen Dipols \cite{web2}.}\label{fig:entstehung}
\end{figure}

Eine Dipol Antenne kann als kurzer Linearstrahler beschrieben werden, wobei dessen Länge $l \ll \lambda_0 /4$ beträgt. $\lambda_0$ ist dabei die Wellenlänge, welche durch das Verhältnis der Lichtgeschwindigkeit zur Signalfrequenz $c_0/f$ beschrieben wird. 

\begin{figure}[H]
	\centering
	\includegraphics[width=0.4\linewidth]{hertzscher_dipol}
	\caption{Hertzscher Dipol im Koordinatensystem.}\label{fig:hertzsche Dipol}
\end{figure}

Ein solcher Strahler ist in Abbildung \ref{fig:hertzsche Dipol} zu sehen, wobei $\underline{I}$ der komplexe Strom einer Sinusschwingung darstellt, welche konstant über die gesamte Länge schwingt. $\varphi$ stellt den Winkel dar, welcher um die z-Achse rotiert und über die x- und y-Achse aufgespannt ist. $\vartheta$ hingegen rotiert um die x-Achse und wird über die y- und z-Achse aufgespannt. Für die Kugelkoordinaten bedeutet dies, dass mit den Restriktionen 

\begin{align}
&0 \leq \varphi \leq 2\pi\\
&0 \leq \vartheta \leq \pi \label{eq:HertzTheta}
\end{align}

der gesamte Bereich des Koordinatensystems abgedeckt ist. Das durch den Strom induzierte Magnetfeld kann mit 

\begin{equation}
\underline{H}_0 = j\pi \frac{\underline{I}l}{{\lambda_0}^2}
\end{equation}

beschrieben werden, was zwei Formeln für das magnetische und elektrische Feld liefert:

\begin{align}
\underline{H}_\varphi   &= \underline{H}_0 \sin \vartheta \frac{e^{-jk_0r}}{k_0r}\\
\underline{E}_\vartheta &= Z_0 \underline{H}_\varphi = Z_0 \underline{H}_0 \sin \vartheta \frac{e^{-jk_0r}}{k_0r}.\label{eq:HertzE}
\end{align}

Hierbei beschreibt $r$ die Distanz des Punktes zum Ursprung und $k_0 = \omega \sqrt{\mu_0 \eta_0} = c_0/\omega$ die Wellenzahl im Vakuum. Somit ist die komplexe Exponentialfunktion die Ausbreitung der Welle, was mit der Theorie der Wellenleitern übereinstimmt. Im Nenner ist das Abklingen des Sinus zu erkennen, welches abhängig von der Distanz zum Ursprung und der Wellenzahl ist.

\subsection{Richtdiagramm}

Ein wichtiges Diagramm für die Analyse von Antennen ist das Richtdiagramm. Mit diesem kann die Direktivität einer Antenne dargestellt werden, welche mitteilt, in welche Richtung wie viel der Leistung der Antenne abgestrahlt wird. Diese Analyse geschieht im Fernfeld, was bedeutet dass die Annahme $r \rightarrow \infty$ getroffen wird. Somit kann verhindert werden, dass eine Antenne in eine unerwünschte Richtung Leistung abgibt und somit eine andere Antenne stören könnte. Daher wird die Feldstärke einer Antenne in Abhängigkeit der Raumrichtung beschreiben, welche mit den Kugelkoordinaten $\varphi$ und $\vartheta$ angegeben wird.\\

Für das Richtdiagramm wird die komplexe elektrische Feldstärke in den Kugelkoordinaten auf dessen maximale Feldstärke normiert, wobei sich dadurch ein wert zwischen $0$ und $1$ einstellt (Simulationsprogramme können eine andere Normierung vornehmen, worauf acht genommen werden sollte). Dies ergibt die auf den Maximalwert normierte Winkelverteilung des elektrischen Feldes. Diese Kenngrösse wird als Richtcharakteristik bezeichnet:

\begin{equation}
C(\vartheta,\varphi) = \frac{|\underline{E}(\vartheta,\varphi)|}{|\underline{E}(\vartheta,\varphi)|_{\mathrm{max}}}.
\end{equation}

Als Beispiel wird der Hertzsche Dipol aus dem Abschnitt \ref{sec:HerDip} betrachtet. Dessen elektrische Feldstärke ist in Formel \ref{eq:HertzE} beschrieben, wobei durch den Sinus dessen Maximalwert direkt bestimmt werden kann. Somit ergibt sich für die Richtcharakteristik:

\begin{equation}\label{eq:HertRicht}
C(\vartheta,\varphi) = \frac{|\underline{E}(\vartheta,\varphi)|}{|\underline{E}(\vartheta,\varphi)|_{\mathrm{max}}} = \frac{\frac{Z_0 |\underline{H}_0| |\sin \vartheta|}{k_0r}}{\frac{Z_0 |\underline{H}_0|}{k_0r}} = \sin \vartheta.
\end{equation}

Aus Formel \ref{eq:HertRicht} ist herauszulesen, dass an der Stellen für $\vartheta$ gleich Null oder $\pi$ eine Nullstelle ist, das Maximum an der Stelle $\pi/2$ mit dem normierten Wert 1 erreicht wird und bei $\pi/4$ der Wert $1/\sqrt{2}$ anliegt.\\

Zur Verifikation wurde zuerst ein MATLAB-Code erstellt, welcher die Polarkoordinaten darstel-len soll (siehe Appendix). Zu erwarten ist dabei ein identischer Plot zu dem im Buch von K. Kark auf Seite 162, welcher in Abbildung \ref{fig:dipol} zu sehen ist \cite{book}.

\begin{figure}[!ht]
	\centering
	\includegraphics[width=\linewidth]{Dipol.png}
	\caption{Vertikal- und Horizontalschnitt eines Hertzschen Dipoles.}\label{fig:dipol}
\end{figure}

\newpage

\begin{figure}[!ht]
	\centering
    \includegraphics[width=\textwidth]{HertzscherDipol.png}
    \caption{Simulationen des Hertzschen Dipoles mit der Anordnung in allen drei verschiedenen Achsenrichtungen.}
    \label{fig:HertzscherDipol}
\end{figure}

Das Resultat ist in Abbildung \ref{fig:HertzscherDipol} zu sehen. Die Werte stimmen mit den Annahmen und der erwarteten Figur \ref{fig:dipol} überein. Zu beachten ist jedoch, dass nur die Hälfte des Kreises ersichtlich wäre, wenn wie in Formel \ref{eq:HertzTheta} definiert $\vartheta$ nur über den halben Bereich gewählt werden würde. Zur einfacheren Darstellung wurde hier aber $\vartheta$ von $0 ... \pi$ gewählt. Bei der Berechnung wird zwischen dem vertikalen und horizontalen Richtdiagramm unterschieden, wobei für das vertikale Diagramm mit $C(\vartheta, \varphi = \varphi_0)$ die Strahlung im Elevationswinkel $\vartheta$ angegeben wird und beim horizontalen Diagramm mit $C(\vartheta=\pi/2, \varphi)$ die Strahlung im sogenannten Azimutwinkel $\varphi$ angegeben wird. Dies resultiert in der E-Ebene für das vertikale Diagramm und in der H-Ebene für das horizontale Diagramm des z-Dipols (vergleichbar mit Abbildung \ref{fig:dipol}). Beim x- und y-Dipol ergibt sich eine identische Abstrahlungseigenschaft, wobei die Richtung der Abstrahlung nicht identisch ist. Dazu muss der Winkel $\varphi_0$ jeweils angepasst werden, damit die korrekten Richtdiagramme dargestellt werden können.\\

Als letzter Teil der Theorie wird auf die Halbwertsbreiten $\Delta \vartheta$ und $\Delta \varphi$ eingegangen. Diese beschreiben die Breite der Hauptkeule, wobei der Winkel unter welchem die Hälfte der Strahlungsdichte liegt angegeben wird (beziehungsweise der -3dB Punkt). Da wie schon erwähnt beim Hertzschen Dipol bei $\pi/4$ die Richtcharakteristik $1/\sqrt{2}$ beträgt, ergibt sich für die Halbwertsbreiten einen Winkel von \SI{90}{\degree}.

\subsection{Simulation}

Für die Simulation wurde eine CST Datei erstellt mit \SI{1}{\giga\hertz} Grenzfrequenz. Die Länge beträgt $\lambda_0/40$ und der Durchmesser ist um den Faktor \num{75} kleiner als die Länge. Für die Anregung in der Mitte wurde ein Vakuum erstellt, in welchem sich der Port zum Erregen befindet.

\begin{figure}[!ht]
	\centering
	\includegraphics[width=\linewidth]{Hertzian.png}
	\caption{Richtdiagramm des Hertzschen Dipols.}\label{fig:Hertzian}
\end{figure}

In Abbildung \ref{fig:Hertzian} ist das Richtdiagramm der Simulation zu finden. Dabei strahlt die Hauptkeule wie berechnet in Richtung \SI{90}{\degree} ab und die Halbwertszahl beträgt \SI{89.9}{\degree}, was nur minimal vom Erwartungswert abweicht.

