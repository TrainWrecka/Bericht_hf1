%%---Main Packages-----------------------------------------------------------------------
\usepackage[english, ngerman]{babel}	%Mul­tilin­gual sup­port for LaTeX
\usepackage[T1]{fontenc}				%Stan­dard pack­age for se­lect­ing font en­cod­ings
\usepackage[utf8]{inputenc}				%Ac­cept dif­fer­ent in­put en­cod­ings
\usepackage{lmodern}                    %The newer Font-Set
\usepackage{textcomp}					%LaTeX sup­port for the Text Com­pan­ion fonts
\usepackage{graphicx} 					%En­hanced sup­port for graph­ics
\usepackage{float}						%Im­proved in­ter­face for float­ing ob­jects
\usepackage{ifdraft}                    %Let you check if the doc is in draft mode

%%---Useful Packages---------------------------------------------------------------------
\usepackage[pdftex,dvipsnames]{xcolor}  %Driver-in­de­pen­dent color ex­ten­sions for LaTeX
\usepackage{csquotes}                   %Simpler quoting with \enquote{}
\usepackage{siunitx} 					%A com­pre­hen­sive (SI) units pack­age
\usepackage{listings}					%Type­set source code list­ings us­ing LaTeX
\usepackage[bottom]{footmisc}			%A range of foot­note op­tions
\usepackage{footnote}					%Im­prove on LaTeX's foot­note han­dling
\usepackage{verbatim}					%Reim­ple­men­ta­tion of and ex­ten­sions to LaTeX ver­ba­tim
\usepackage[textsize=footnotesize]{todonotes} %Mark­ing things to do in a LaTeX doc­u­ment

%%---Tikz Packages-----------------------------------------------------------------------
\usepackage{standalone}
\usepackage{tikz}
\usepackage{circuitikz}
\usetikzlibrary{arrows}
\usetikzlibrary{calc}
\usetikzlibrary{intersections}

%%---Math Packages-----------------------------------------------------------------------
\usepackage{amsmath}					%AMS math­e­mat­i­cal fa­cil­i­ties for LaTeX
%\usepackage{amssymb}					%Type­set­ting symbols (AMS style)
%\usepackage{array}						%Ex­tend­ing the ar­ray and tab­u­lar en­vi­ron­ments
%\usepackage{amsthm}					%Type­set­ting the­o­rems (AMS style)

%%---Table Packages----------------------------------------------------------------------
\usepackage{tabularx}					%Tab­u­lars with ad­justable-width columns
%\usepackage{longtable}
\usepackage{multirow}					%Create tab­u­lar cells span­ning mul­ti­ple rows
\usepackage{multicol}					%In­ter­mix sin­gle and mul­ti­ple columns

%%---PDF / Figure Packages---------------------------------------------------------------
\usepackage{pdfpages}					%In­clude PDF doc­u­ments in LaTeX
\usepackage{pdflscape}					%Make land­scape pages dis­play as land­scape
\usepackage{subfig}					    %Fig­ures di­vided into sub­fig­ures
\usepackage{placeins}                   % use \FloatBarrier to restrict float behind this place

%%---Other Packages----------------------------------------------------------------------
%\usepackage{xargs}                     %De­fine com­mands with many op­tional ar­gu­ments

%%---Bibliography------------------------------------------------------------------------
\usepackage[style=ieee,urldate=comp,backend=biber]{biblatex}
\addbibresource{literature/bibliography.bib}

%%---Main Settings-----------------------------------------------------------------------
\graphicspath{{./graphics/}}			%Defines the graphicspath
%\geometry{twoside=false}				    %twoside=false disables the "bookstyle"
\setlength{\marginparwidth}{2cm}
\overfullrule=5em						%Creates a black rule if text goes over the margins => debugging


%%---User Definitions--------------------------------------------------------------------
%%Tabel-Definitions: (requires \usepackage{tabularx})
\newcolumntype{L}[1]{>{\raggedright\arraybackslash}p{#1}}    %column-width and alignment
\newcolumntype{C}[1]{>{\centering\arraybackslash}p{#1}}
\newcolumntype{R}[1]{>{\raggedleft\arraybackslash}p{#1}}

%%---Optional Package Settings-----------------------------------------------------------
%Listings-Settings: (requires \usepackage{listings}) => Example with Matlab Code
\lstset{language=Matlab,%
    basicstyle=\footnotesize\ttfamily,
    breaklines=false,%
    morekeywords={switch, case, otherwise},
    keywordstyle=\color{Blue},%
    tabsize=2,
    %morekeywords=[2]{1}, keywordstyle=[2]{\color{black}},
    identifierstyle=\color{Black},%
    stringstyle=\color{Purple},
    commentstyle=\color{Green},%
    showstringspaces=false,%without this there will be a symbol in the places where there is a space
    numbers=left,%
    numberstyle={\tiny \color{black}},% size of the numbers
    numbersep=9pt, % this defines how far the numbers are from the text
    %emph=[1]{word1, word2,...},emphstyle=[1]\color{red}
}				

%%---Projectspecific------------------------------------------------------------------------
\usepackage{pgfplots}	
%\usepackage{IEEEtrantools}	
%\usepackage{array}
%\usepackage{lipsum}
%\usepackage{etoolbox}
%\usepackage{setspace}
\usetikzlibrary{shapes,decorations.markings,backgrounds,patterns}
\usepackage[framed,numbered,autolinebreaks,useliterate]{mcode}

%%%
%%% For the SFGs
%%%
\tikzset{%
% Style of the node
    Node/.style={circle,thick,draw=black,inner sep=0, minimum size=0.15cm},
    Start/.style={draw=red},
    End/.style={draw=blue},
    Interm/.style={},
% Style of the node label
    NodeName/.style={font=\footnotesize,black, outer sep=1},
    NodeName n/.style={NodeName, above},
    NodeName s/.style={NodeName, below},
    NodeName e/.style={NodeName, right},
    NodeName w/.style={NodeName, left},
% Style of the branche label
    ArrowName/.style={font=\footnotesize,auto,outer sep=1},
    ArrowName n/.style={ArrowName, above},
    ArrowName s/.style={ArrowName, below},
    ArrowName e/.style={ArrowName, right},
    ArrowName w/.style={ArrowName, left},
% Style of the branch
    Connection/.style={thick},
    NodeBezier/.style={},
    ->-/.style={decoration={
        markings,
        %mark=at position #1 with {\arrow[scale=1.3,shorten >=1cm]{>}}},
        mark=at position #1 with {\draw[->,>=latex',ultra thick](0pt,0)--(4pt,0);}},
        postaction={decorate}},
    ->-/.default=0.50,
}

% Engineering

\makeatletter

\newif\ifpgfplots@scaled@x@ticks@engineering
\pgfplots@scaled@x@ticks@engineeringfalse
\newif\ifpgfplots@scaled@y@ticks@engineering
\pgfplots@scaled@y@ticks@engineeringfalse
\newif\ifpgfplots@scaled@z@ticks@engineering
\pgfplots@scaled@z@ticks@engineeringfalse

\pgfplotsset{
    scaled x ticks/engineering/.code=
        \pgfplots@scaled@x@ticks@engineeringtrue,
    scaled y ticks/engineering/.code=
        \pgfplots@scaled@y@ticks@engineeringtrue,
    scaled z ticks/engineering/.code=
        \pgfplots@scaled@y@ticks@engineeringtrue,
%    scaled ticks=engineering  % Uncomment this line if you want "engineering" to be on by default
}

\def\pgfplots@init@scaled@tick@for#1{%
    \global\def\pgfplots@glob@TMPa{0}%
    \expandafter\pgfplotslistcheckempty\csname pgfplots@prepared@tick@positions@major@#1\endcsname
    \ifpgfplotslistempty
        % we have no tick labels. Omit the tick scale label as well!
    \else
    \begingroup
    \ifcase\csname pgfplots@scaled@ticks@#1@choice\endcsname\relax
    % CASE 0 : scaled #1 ticks=false: do nothing here.
    \or
        % CASE 1 : scaled #1 ticks=true:
        %--------------------------------
        % the \pgfplots@xmin@unscaled@as@float  is set just before the data
        % scale transformation is initialised.
        %
        % The variables are empty if there is no datascale transformation.
        \expandafter\let\expandafter\pgfplots@cur@min@unscaled\csname pgfplots@#1min@unscaled@as@float\endcsname
        \expandafter\let\expandafter\pgfplots@cur@max@unscaled\csname pgfplots@#1max@unscaled@as@float\endcsname
        %
        \ifx\pgfplots@cur@min@unscaled\pgfutil@empty
            \edef\pgfplots@loc@TMPa{\csname pgfplots@#1min\endcsname}%
            \expandafter\pgfmathfloatparsenumber\expandafter{\pgfplots@loc@TMPa}%
            \let\pgfplots@cur@min@unscaled=\pgfmathresult
            \edef\pgfplots@loc@TMPa{\csname pgfplots@#1max\endcsname}%
            \expandafter\pgfmathfloatparsenumber\expandafter{\pgfplots@loc@TMPa}%
            \let\pgfplots@cur@max@unscaled=\pgfmathresult
        \fi
        %
        \expandafter\pgfmathfloat@decompose@E\pgfplots@cur@min@unscaled\relax\pgfmathfloat@a@E
        \expandafter\pgfmathfloat@decompose@E\pgfplots@cur@max@unscaled\relax\pgfmathfloat@b@E
        \pgfplots@init@scaled@tick@normalize@exponents
        \ifnum\pgfmathfloat@b@E<\pgfmathfloat@a@E
            \pgfmathfloat@b@E=\pgfmathfloat@a@E
        \fi
        \xdef\pgfplots@glob@TMPa{\pgfplots@scale@ticks@above@exponent}%
        \ifnum\pgfplots@glob@TMPa<\pgfmathfloat@b@E
            % ok, scale it:
            \expandafter\ifx % Check whether we're using engineering notation (restricting exponents to multiples of three)
                \csname ifpgfplots@scaled@#1@ticks@engineering\expandafter\endcsname
                \csname iftrue\endcsname
                    \divide\pgfmathfloat@b@E by 3
                    \multiply\pgfmathfloat@b@E by 3
            \fi
            \multiply\pgfmathfloat@b@E by-1
            \xdef\pgfplots@glob@TMPa{\the\pgfmathfloat@b@E}%
        \else
            \xdef\pgfplots@glob@TMPa{\pgfplots@scale@ticks@below@exponent}%
            \ifnum\pgfplots@glob@TMPa>\pgfmathfloat@b@E
                % ok, scale it:
                \expandafter\ifx % Check whether we're using engineering notation (restricting exponents to multiples of three)
                    \csname ifpgfplots@scaled@#1@ticks@engineering\expandafter\endcsname
                    \csname iftrue\endcsname
                        \advance\pgfmathfloat@b@E by -2
                        \divide\pgfmathfloat@b@E by 3
                        \multiply\pgfmathfloat@b@E by 3
                \fi
                \multiply\pgfmathfloat@b@E by-1
                \xdef\pgfplots@glob@TMPa{\the\pgfmathfloat@b@E}%
            \else
                % no scaling necessary:
                \xdef\pgfplots@glob@TMPa{0}%
            \fi
        \fi
    \or
        % CASE 2 : scaled #1 ticks=base 10:
        %--------------------------------
        \c@pgf@counta=\csname pgfplots@scaled@ticks@#1@arg\endcsname\relax
        %\multiply\c@pgf@counta by-1
        \xdef\pgfplots@glob@TMPa{\the\c@pgf@counta}%
    \or
        % CASE 3 : scaled #1 ticks=real:
        %--------------------------------
        \pgfmathfloatparsenumber{\csname pgfplots@scaled@ticks@#1@arg\endcsname}%
        \global\let\pgfplots@glob@TMPa=\pgfmathresult
    \or
        % CASE 4 : scaled #1 ticks=manual:
        \expandafter\global\expandafter\let\expandafter\pgfplots@glob@TMPa\csname pgfplots@scaled@ticks@#1@arg\endcsname
    \fi
    \endgroup
    \fi
    \expandafter\let\csname pgfplots@tick@scale@#1\endcsname=\pgfplots@glob@TMPa%
}
\makeatother